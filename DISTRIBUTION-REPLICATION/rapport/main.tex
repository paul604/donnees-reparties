\documentclass{article}
\usepackage[utf8]{inputenc}
\usepackage[frenchb]{babel}
\usepackage[T1]{fontenc}
\usepackage{authblk}
\usepackage{hyperref}
\usepackage{fancyhdr}
\usepackage{titling}
\usepackage{graphicx}
\usepackage{geometry}
\usepackage{enumitem}
\usepackage{microtype}
\usepackage[none]{hyphenat}

 \geometry{
 a4paper,
 total={169mm,240mm},
 left=16mm,
 top=20mm,
 }

\usepackage{listings} % For code coloration
\usepackage{color}
\usepackage[dvipsnames]{xcolor}


\definecolor{codegreen}{rgb}{0,0.6,0}
\definecolor{codecodegray}{rgb}{0.5,0.5,0.5}
\definecolor{codepurple}{rgb}{0.58,0,0.82}
\definecolor{backcolour}{rgb}{0.95,0.95,0.92}
\definecolor{darkblue}{rgb}{0.0,0.0,0.6}
\definecolor{cyan}{rgb}{0.0,0.0,0.6}



% \headheight = 14pt

\hypersetup{colorlinks = true,citecolor=black,filecolor=black,linkcolor=black,urlcolor=black}

\lstdefinestyle{s}{
  backgroundcolor=\color{backcolour},   commentstyle=\color{codegreen},
  keywordstyle=\color{NavyBlue},
  numberstyle=\tiny\color{codecodegray},
  stringstyle=\color{codepurple},
  basicstyle=\footnotesize,
  breakatwhitespace=false,         
  breaklines=true,                 
  captionpos=b,                    
  keepspaces=true,                 
  numbers=left,                    
  numbersep=5pt,                  
  showspaces=false,                
  showstringspaces=false,
  showtabs=false,                  
  tabsize=4
}


\lstset{style=s}
\lstdefinelanguage{XML}
{  
  breakatwhitespace=false,         
  breaklines=true,
  morestring=[b]",
  morestring=[s]{>}{<},
  morecomment=[s]{<?}{?>},
  stringstyle=\color{black},
  identifierstyle=\color{cyan},
  keywordstyle=\color{darkblue},
  morekeywords={xmlns,version,type}
}

\lstset{literate=
  {á}{{\'a}}1 {é}{{\'e}}1 {í}{{\'i}}1 {ó}{{\'o}}1 {ú}{{\'u}}1
  {Á}{{\'A}}1 {É}{{\'E}}1 {Í}{{\'I}}1 {Ó}{{\'O}}1 {Ú}{{\'U}}1
  {à}{{\`a}}1 {è}{{\`e}}1 {ì}{{\`i}}1 {ò}{{\`o}}1 {ù}{{\`u}}1
  {À}{{\`A}}1 {È}{{\'E}}1 {Ì}{{\`I}}1 {Ò}{{\`O}}1 {Ù}{{\`U}}1
  {ä}{{\"a}}1 {ë}{{\"e}}1 {ï}{{\"i}}1 {ö}{{\"o}}1 {ü}{{\"u}}1
  {Ä}{{\"A}}1 {Ë}{{\"E}}1 {Ï}{{\"I}}1 {Ö}{{\"O}}1 {Ü}{{\"U}}1
  {â}{{\^a}}1 {ê}{{\^e}}1 {î}{{\^i}}1 {ô}{{\^o}}1 {û}{{\^u}}1
  {Â}{{\^A}}1 {Ê}{{\^E}}1 {Î}{{\^I}}1 {Ô}{{\^O}}1 {Û}{{\^U}}1
  {œ}{{\oe}}1 {Œ}{{\OE}}1 {æ}{{\ae}}1 {Æ}{{\AE}}1 {ß}{{\ss}}1
  {ű}{{\H{u}}}1 {Ű}{{\H{U}}}1 {ő}{{\H{o}}}1 {Ő}{{\H{O}}}1
  {ç}{{\c c}}1 {Ç}{{\c C}}1 {ø}{{\o}}1 {å}{{\r a}}1 {Å}{{\r A}}1
  {€}{{\EUR}}1 {£}{{\pounds}}1 {°}{{\no}}1
}



\pretitle{ 
  \begin{center}
  \includegraphics[width=60mm,height=31mm]{univ.png}
  \qquad \qquad
  \includegraphics[width=37mm,height=31mm]{iutNantes.jpg}\\[\bigskipamount]
}
\posttitle{\end{center}}

\title{\huge{TD Distribution – Replication}\\
\normalsize Donnée Repartie}
\date{\today}
\author{Paul Orhon \\
\small LP -- MiAR -- Université de Nantes }

\pagestyle{fancy}
\fancyhf{}
\rhead{Paul Orhon --- \small LP -- MiAR}
\lhead{TD Distribution – Replication --- Donnée Repartie}
\rfoot{Page \thepage}
\lfoot{INSTITUT UNIVERSITAIRE DE TECHNOLOGIE - NANTES}

\begin{document}

\maketitle
\tableofcontents

\clearpage

\section{Notre bibliothèque s'agrandit}
\subsection{Question 1}
Les tables requises sur DB2 sont:
\begin{itemize}
    \item Emprunt
    \item Exemplaire
    \item ExemplaireEmprunte
\end{itemize}
Pour le détail voir le fichier \href{https://github.com/paul604/donnees-reparties/blob/master/DISTRIBUTION-REPLICATION/CreationServeur2.sql}{CreationServeur2.sql}
\\

Pour éviter des problème de cohérence des donnée il faut verrouiller les table présente sur les deux bd avant un \textbf{insert} ou un \textbf{update}.
\begin{lstlisting}[language=SQL, caption= Trigger lock Exemplaire]
create or replace TRIGGER Lock_Exemplaire_Ins_Up
BEFORE INSERT OR UPDATE ON Exemplaire 
BEGIN
  LOCK TABLE Exemplaire IN SHARE MODE;
  LOCK TABLE Exemplaire@vers_Angers IN SHARE MODE;
END;
\end{lstlisting}

\subsection{Question 2}
Pour que le champ \textbf{ISBN} de la table \textbf{EXEMPLAIRE} de la BD2 référence au champ \textbf{ISBN} de la table \textbf{LIVRE} de la BD1 il faut ajouter un trigger.

\label{lst:1}
\begin{lstlisting}[language=SQL, caption= Trigger]
CREATE OR replace TRIGGER Isbn_ex_ref_livre_to_BD1
    AFTER INSERT OR UPDATE on EXEMPLAIRE
    DECLARE
    rec_ex EXEMPLAIRE%rowtype;
begin
    SELECT * INTO rec_ex
        FROM EXEMPLAIRE 
        WHERE ISBN not IN
            (SELECT LIVRE.ISBN FROM LIVRE@vers_Nantes);
    raise_application_error (-20202,'viol de fk ISBN'); 
EXCEPTION
    WHEN no_data_found then NULL;
    WHEN too_many_rows then
        raise_application_error (-20202,'viol de fk ISBN');
END;
\end{lstlisting}

Ce trigger permet, apprêt un \textbf{insert} ou un \textbf{update} de la table \textbf{EXEMPLAIRE} de BD2, de vérifier la présence des \textbf{ISBN} de cette table dans la table \textbf{LIVRE} de BD1. Si le \textbf{selecte} %ligne???%
trouve 0 enregistrement il n'y a pas de problème, mais si il trouve 1 ou plusieurs enregistrement il lève une exception. 
\\

Pour vérifier que le champ \textbf{numA} de la table \textbf{EMPRUNT} du serveur BD2, référence le champ \textbf{numA} de la table \textbf{ADHERENT} du serveur BD1 on met en place un trigger identique au précédant (\ref{lst:1}) mais en replacent \textbf{EXEMPLAIRE} par \textbf{EMPRUNT} et \textbf{LIVRE} par \textbf{ADHERENT} et en changent le deuxième selecte par \textit{ADHERENT.NUMA}.
\\

Il faut aussi créer un trigger pour verrouiller les table lors de changement.
Avant un \textbf{insert} ou un \textbf{update} sur \textbf{EXEMPLAIRE} du serveur BD2 il faut verrouiller la table \textbf{LIVRE}:
\begin{lstlisting}[language=SQL, caption= Trigger]
CREATE OR replace TRIGGER Lock_Livre_toBD1
    BEFORE INSERT OR UPDATE ON EXEMPLAIRE
BEGIN
    lock table LIVRE@vers_Nantes in share mode;
end;
\end{lstlisting}
Et pour un \textbf{insert} ou un \textbf{update} sur la table \textbf{EMPRUNT} du serveur BD2 il faut verrouiller la table \textbf{ADHERENT}.


\subsection{Question 3}
Pour que le champ \textbf{ISBN} de la table \textbf{LIVRE} de la BD1 référence au champ \textbf{ISBN} de la table \textbf{EXEMPLAIRE} de la BD2 il faut ajouter un trigger.

\label{lst:2}
\begin{lstlisting}[language=SQL, caption= Trigger]
CREATE OR replace TRIGGER  Isbn_livre_isRefBy_ex_to_BD2
    AFTER DELETE ON LIVRE
    DECLARE
    rec_ex EXEMPLAIRE@vers_Angers%rowtype;
begin
    SELECT * INTO rec_ex 
        FROM EXEMPLAIRE@vers_Angers
        WHERE ISBN not IN
            (SELECT LIVRE.ISBN FROM LIVRE);
    raise_application_error (-20202,'viol de fk ISBN. Enfant dans BD2'); 
EXCEPTION
    WHEN no_data_found then NULL;
    WHEN too_many_rows then
        raise_application_error (-20202,'viol de fk ISBN. Enfants dans BD2');
END;
\end{lstlisting}

Ce trigger permet, avant un \textbf{delete} sur la table \textbf{LIVRE} de BD1, de vérifier la présence des \textbf{ISBN} de cette table dans la table \textbf{EXEMPLAIRE} de BD2. Si le \textbf{selecte} %ligne???%
trouve 0 enregistrement il n'y a pas de problème, mais si il trouve 1 ou plusieurs enregistrement il lève une exception. 
\\

Pour vérifier que le champ \textbf{numA} de la table \textbf{ADHERENT} du serveur BD1, référence le champ \textbf{numA} de la table \textbf{EMPRUNT} du serveur BD2 on met en place un trigger identique au précédant (\ref{lst:2}) mais en replacent \textbf{LIVRE} par \textbf{ADHERENT} et \textbf{EXEMPLAIRE} par \textbf{EMPRUNT} et en changent le deuxième selecte par \textit{ADHERENT.NUMA}.
\\

Il faut aussi créer un trigger pour verrouiller les table lors de changement.
Avant un \textbf{delete} sur \textbf{LIVRE} il faut verrouiller la table \textbf{EXEMPLAIRE} du serveur BD2:
\begin{lstlisting}[language=SQL, caption= Trigger]
CREATE OR replace TRIGGER Lock_ex_toBD2
BEFORE DELETE ON LIVRE
BEGIN
lock table EXEMPLAIRE@vers_Angers in share mode;
end;
\end{lstlisting}
Et pour un \textbf{delete} sur la table \textbf{ADHERENT} il faut verrouiller la table \textbf{EMPRUNT} du serveur BD2.


\subsection{Question 4}
Pour mettre à jour le compteur, la date et le lieu d'emprunt
dans la table \textbf{LIVRE} de BD1, on peur copier et modifier le trigger déjà présent dans BD1.
\begin{lstlisting}[language=SQL, caption= Trigger up\_val\_livre\_DB1]
create or replace trigger up_val_livre_DB1
    AFTER INSERT ON EXEMPLAIREEMPRUNTE for each row
BEGIN
    UPDATE LIVRE@vers_Nantes
        SET NBE = NBE+1,
            DATEDE = :NEW.dateE,
            LIEUDE = 'Angers'
        WHERE ISBN = (
            SELECT ISBN 
            FROM EXEMPLAIRE 
            where EXEMPLAIRE.NUMINV = :NEW.numInv 
        );
END;
\end{lstlisting}


\subsection{Question 5}
Pour cette question il ne faut pas oublier les triggers qui lock les table \textbf{EXEMPLAIREEMPRUNTE} dans les deux serveur pour éviter les problèmes.
Ils ont été réaliser à la question 1.
\\

Ensuite on reprend le trigger déjà réaliser sur le serveur 1 qui permet de ne pas emprunter plus de 5 exemplaire que l'on modifie et que l'on place sur les deux serveur, en remodifiant les @vers\_*.
\begin{lstlisting}[language=SQL, caption= Trigger Max exemplaire DB2]
create or replace TRIGGER max_exem_emp
    AFTER INSERT on EXEMPLAIREEMPRUNTE
    declare
        ligne_exem_emp ADHERENT@vers_Nantes%rowtype;
begin
    SELECT * into ligne_exem_emp FROM ADHERENT@vers_Nantes 
        WHERE (
            (SELECT COUNT(*) FROM EXEMPLAIREEMPRUNTE exN
                WHERE ADHERENT.NUMA=exN.NUMA)
            +
            (SELECT COUNT(*) FROM EXEMPLAIREEMPRUNTE@vers_Nantes exA
                WHERE ADHERENT.NUMA=exA.NUMA)
        )>5;
    raise_application_error (-20001, 'un adhérant a emprunté plus de 5 exemplaires');
    exception
        when no_data_found then null;
        when too_many_rows then
    raise_application_error (-20002, 'des adhérants ont emprunté plus de 5 exemplaires');
end;
\end{lstlisting}

\subsection{Question 6}
Sur BD1 il faut seulement modifier le package \textit{Lecture} pour prendre en compte BD2. Pour le package \textit{Maj} il n'est pas utile de prendre en compte BD2 car les mise à jour des table doive s'effectuer seulement sur la base actuelle.
\\

Sur BD2 il faut mètre à jour le package \textit{Lecture} pour prendre en compte BD1, et le package \textit{Maj} pour réaliser les modification utile sur BD1 (la création d'adhérant).
\begin{lstlisting}[language=SQL, caption= Un procédure de \textit{Lecture} sur BD1]
PROCEDURE listEmpruntsForAdherent(idAdh NUMBER, liste out liste_Cursor) IS
    BEGIN
        SET TRANSACTION READ ONLY;
        OPEN liste FOR select * from(
            SELECT Exemplaire.NUMINV, Livre.titre
                FROM Livre, EXEMPLAIRE
                WHERE Livre.ISBN = EXEMPLAIRE.ISBN
                    AND EXEMPLAIRE.numInv IN (
                        SELECT numInv
                        FROM ExemplaireEmprunte
                        WHERE numA = idAdh
                    )

            Union all

            SELECT exA.NUMINV, Livre.titre
                FROM Livre, EXEMPLAIRE@vers_Angers exA
                WHERE Livre.ISBN = exA.ISBN
                    AND exA.numInv IN (
                        SELECT numInv
                        FROM ExemplaireEmprunte@vers_Angers
                        WHERE numA = idAdh
                    )
        );
        COMMIT;
    END;
\end{lstlisting}

\section{La centralisation des adhérents}
\subsection{Question 7}
Les principaux problèmes posés par le partage d'une table en lecture/écriture est la cohérence des données et la mise à jour de ces données. Par exemple si un adhérant avec un \textbf{numA} à 5 est créer sur la table de BD1 et de BD2 on ne sait pas le quelle prendre en compte l'or de la mise en commun des données.

\subsubsection{Question 8}
Pour réaliser le partage en lecture seule de la table adhérent de BD1 vers BD2, il faut mètre en place une table de log d'adhérent.
\begin{lstlisting}[language=SQL, caption= Log d'adhérant sur BD1]
CREATE MATERIALIZED view log on Adherent;
\end{lstlisting}
Puis sur BD2 il faut créer la vue qui va, toute les minute, récupérer les mise à jour présente dans les log de la table adhérant.
\begin{lstlisting}[language=SQL, caption= View d'adhérant sur BD2]
CREATE MATERIALIZED view Adherent_view
    refresh force
    start with sysdate
    next sysdate+1/1440
    as select * from Adherent;
\end{lstlisting}

\subsubsection{Question 9}
Il suffi de prendre le trigger existant et d'ajouter un select vers la table \textbf{exemplaireEmprunte} de l'autre serveur.

\begin{lstlisting}[language=SQL, caption= Trigger max exemplaire emprunter version BD1]
create or replace TRIGGER max_exem_emp
    AFTER INSERT on EXEMPLAIREEMPRUNTE
    declare
        ligne_exem_emp ADHERENT%rowtype;
begin
    SELECT * into ligne_exem_emp FROM ADHERENT 
        WHERE (
            (SELECT COUNT(*) FROM EXEMPLAIREEMPRUNTE exN
                WHERE ADHERENT.NUMA=exN.NUMA)
            +
            (SELECT COUNT(*) FROM EXEMPLAIREEMPRUNTE@vers_Angers exA
                WHERE ADHERENT.NUMA=exA.NUMA)
        )>5;
    raise_application_error (-20001, 'un adhérant a emprunté plus de 5 exemplaires');
    exception
        when no_data_found then null;
        when too_many_rows then
    raise_application_error (-20002, 'des adhérants ont emprunté plus de 5 exemplaires');
end;
\end{lstlisting}

\end{document}
