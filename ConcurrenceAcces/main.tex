\documentclass{article}
\usepackage[utf8]{inputenc}
\usepackage[frenchb]{babel}
\usepackage[T1]{fontenc}
\usepackage{authblk}
\usepackage{hyperref}
\usepackage{fancyhdr}
\usepackage{titling}
\usepackage{graphicx}
\usepackage{geometry}
\usepackage{enumitem}
\usepackage{microtype}
\usepackage[none]{hyphenat}

 \geometry{
 a4paper,
 total={169mm,240mm},
 left=16mm,
 top=20mm,
 }

\usepackage{listings} % For code coloration
\usepackage{color}
\usepackage[dvipsnames]{xcolor}


\definecolor{codegreen}{rgb}{0,0.6,0}
\definecolor{codecodegray}{rgb}{0.5,0.5,0.5}
\definecolor{codepurple}{rgb}{0.58,0,0.82}
\definecolor{backcolour}{rgb}{0.95,0.95,0.92}
\definecolor{darkblue}{rgb}{0.0,0.0,0.6}
\definecolor{cyan}{rgb}{0.0,0.0,0.6}



% \headheight = 14pt

\hypersetup{colorlinks = true,citecolor=black,filecolor=black,linkcolor=black,urlcolor=black}

\lstdefinestyle{s}{
  backgroundcolor=\color{backcolour},   commentstyle=\color{codegreen},
  keywordstyle=\color{NavyBlue},
  numberstyle=\tiny\color{codecodegray},
  stringstyle=\color{codepurple},
  basicstyle=\footnotesize,
  breakatwhitespace=false,         
  breaklines=true,                 
  captionpos=b,                    
  keepspaces=true,                 
  numbers=left,                    
  numbersep=5pt,                  
  showspaces=false,                
  showstringspaces=false,
  showtabs=false,                  
  tabsize=4
}


\lstset{style=s}
\lstdefinelanguage{XML}
{  
  breakatwhitespace=false,         
  breaklines=true,
  morestring=[b]",
  morestring=[s]{>}{<},
  morecomment=[s]{<?}{?>},
  stringstyle=\color{black},
  identifierstyle=\color{cyan},
  keywordstyle=\color{darkblue},
  morekeywords={xmlns,version,type}
}

\lstset{literate=
  {á}{{\'a}}1 {é}{{\'e}}1 {í}{{\'i}}1 {ó}{{\'o}}1 {ú}{{\'u}}1
  {Á}{{\'A}}1 {É}{{\'E}}1 {Í}{{\'I}}1 {Ó}{{\'O}}1 {Ú}{{\'U}}1
  {à}{{\`a}}1 {è}{{\`e}}1 {ì}{{\`i}}1 {ò}{{\`o}}1 {ù}{{\`u}}1
  {À}{{\`A}}1 {È}{{\'E}}1 {Ì}{{\`I}}1 {Ò}{{\`O}}1 {Ù}{{\`U}}1
  {ä}{{\"a}}1 {ë}{{\"e}}1 {ï}{{\"i}}1 {ö}{{\"o}}1 {ü}{{\"u}}1
  {Ä}{{\"A}}1 {Ë}{{\"E}}1 {Ï}{{\"I}}1 {Ö}{{\"O}}1 {Ü}{{\"U}}1
  {â}{{\^a}}1 {ê}{{\^e}}1 {î}{{\^i}}1 {ô}{{\^o}}1 {û}{{\^u}}1
  {Â}{{\^A}}1 {Ê}{{\^E}}1 {Î}{{\^I}}1 {Ô}{{\^O}}1 {Û}{{\^U}}1
  {œ}{{\oe}}1 {Œ}{{\OE}}1 {æ}{{\ae}}1 {Æ}{{\AE}}1 {ß}{{\ss}}1
  {ű}{{\H{u}}}1 {Ű}{{\H{U}}}1 {ő}{{\H{o}}}1 {Ő}{{\H{O}}}1
  {ç}{{\c c}}1 {Ç}{{\c C}}1 {ø}{{\o}}1 {å}{{\r a}}1 {Å}{{\r A}}1
  {€}{{\EUR}}1 {£}{{\pounds}}1 {°}{{\no}}1
}



\pretitle{ 
  \begin{center}
  \includegraphics[width=60mm,height=31mm]{univ.png}
  \qquad \qquad
  \includegraphics[width=37mm,height=31mm]{iutNantes.jpg}\\[\bigskipamount]
}
\posttitle{\end{center}}

\title{\huge{Concurrence d’accès}\\
\normalsize Donnée Repartie}
\date{\today}
\author{Paul Orhon \\
\small LP -- MiAR -- Université de Nantes }

\pagestyle{fancy}
\fancyhf{}
\rhead{Paul Orhon --- \small LP -- MiAR}
\lhead{Concurrence d’accès --- Donnée Repartie}
\rfoot{Page \thepage}
\lfoot{INSTITUT UNIVERSITAIRE DE TECHNOLOGIE - NANTES}

\begin{document}

\maketitle
\tableofcontents

\clearpage

\section{LECTURE/ECRITURE}
\subsection{Fonctionnement par défaut}
\subsubsection{Que voit-on?}

On observe que l'ajout de l'employer n'est pas visible sur les deux premier \textbf{SELECT} de la première fenêtre. 
\begin{lstlisting}[caption= Premier et deuxième select * from employe;]
    NUEMPL NOMEMPL                   HEBDO     AFFECT
---------- -------------------- ---------- ----------
        .. ...                          ..          .
        65 simone                       30          5
        67 bertrand                     35          5
        73 germaine                     30          5

21 lignes sélectionnées. 
\end{lstlisting}

Mais le nouveau employer est visible sur le dernier, apprêt le commit de la deuxième fenêtre.
\begin{lstlisting}[caption= Dernier select * from employe;]
    NUEMPL NOMEMPL                   HEBDO     AFFECT
---------- -------------------- ---------- ----------
        .. ...                          ..          .
        67 bertrand                     35          5
        73 germaine                     30          5
        99 steve                        20          1
    
22 lignes sélectionnées. 
\end{lstlisting}


\subsubsection{On ne voit pas toujours la même chose dans une transaction}

L'employer que l'on veux supprimer est toujours visible sur les deux premier \textbf{SELECT}.
\begin{lstlisting}[caption= Premier et deuxième select * from employe;]
    NUEMPL NOMEMPL                   HEBDO     AFFECT
---------- -------------------- ---------- ----------
        .. ...                          ..          .
        67 bertrand                     35          5
        73 germaine                     30          5
        99 steve                        20          1
    
22 lignes sélectionnées. 
\end{lstlisting}

Mais le n'ai plus apprêt le commit de la deuxième fenêtre.
\begin{lstlisting}[caption= Dernier select * from employe;]
    NUEMPL NOMEMPL                   HEBDO     AFFECT
---------- -------------------- ---------- ----------
        .. ...                          ..          .
        65 simone                       30          5
        67 bertrand                     35          5
        73 germaine                     30          5

21 lignes sélectionnées. 
\end{lstlisting}


\subsection{Transaction read only}
\subsubsection{On voit toujours la même chose}

Durant toute la duré du \textbf{"set transaction read only;"}, dans la première fenêtre, tout les \textbf{SELECT} affiche la même chose peut importe si la deuxième fenêtre réalise des commit entre le \textbf{"set transaction read only;"} et le \textbf{commit} de la première fenêtre.

\begin{lstlisting}[caption= select * from employe;]
    NUEMPL NOMEMPL                   HEBDO     AFFECT
---------- -------------------- ---------- ----------
        .. ...                          ..          .
        65 simone                       30          5
        67 bertrand                     35          5
        73 germaine                     30          5

21 lignes sélectionnées. 
\end{lstlisting}

\section{ECRITURE/ECRITURE}
\subsection{Fonctionnement par défaut}
\subsubsection{Parfois on croit faire quelque chose et on ne fait rien}
L'or de l'\textbf{UPDATE} de la fenêtre 2 le message de retour est \textit{0 lignes mis à jour}, car la fenêtre 1 a créer l'employer \textit{99} seulement pour lui jusqu'à se qui réalise un commit.
\\
Résultat la requête \textbf{"update employe set hebdo=0 where nuempl=99;"} na rient modifier et la ligne est toujours celle ajouter.
\begin{lstlisting}[caption= select * from employe;]
    NUEMPL NOMEMPL                   HEBDO     AFFECT
---------- -------------------- ---------- ----------
        .. ...                          ..          .
        73 germaine                     30          5
        99 steve                        20          1
\end{lstlisting}

\subsubsection{Ça verrouille}
Le \textbf{delete} de la fenêtre 1 demande un verrou \textit{row exclusive} (RX) sur la table et un verrou \textit{exclusive} (X) sur les lignes. C'est pourquoi l'\textbf{UPDATE} de la fenêtre 2 est mis en attente jusqu'à l'exécution du commit de la fenêtre 1. Quand l'\textbf{UPDATE} s'exécute finalement, il met à jour 0 ligne. Et le \textbf{SELECT} final retourne 0 ligne.

\subsubsection{Parfois ça coince}
Il y a un interblocage.
\\
La fenêtre 1 réalise un \textbf{UPDATE} sur l'employer \no23, donc pose un verrou.
\\
La fenêtre 2 réalise un \textbf{UPDATE} sur l'employer \no39, et pose un verrou.
\\
La fenêtre 2 veut réalise un \textbf{UPDATE} sur l'employer \no23, mais il est verrouiller par la fenêtre 1. Il est mit en attente.
\\
La fenêtre 1 veut réalise un \textbf{UPDATE} sur l'employer \no39, mais il est verrouiller par la fenêtre 2. Il est mit en attente.
\\
A ce moment là, oracle détecte l'interblocage et annule une transaction.
\\
\textit{ORA-00060: détection d'interblocage pendant l'attente d'une ressource}

\subsubsection{Avec les contraintes de type références il y a un problème}
On peut observer qu'a la suite des deux \textbf{commit} les contraintes ne sont pas respecter.
\begin{lstlisting}[caption= select * from employe where nuempl=17;]
aucune ligne sélectionnée
\end{lstlisting}

\begin{lstlisting}[caption= select * from travail where nuempl=17;]
    NUEMPL     NUPROJ      DUREE
---------- ---------- ----------
        17        135          5 
\end{lstlisting}

\subsubsection{Si la contrainte est décrite Oracle pose des verrous et ça marche}


\end{document}
